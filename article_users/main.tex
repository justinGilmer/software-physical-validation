%%%%%%%%%%%%%%%%%%%%%%%%%%%%%%%%%%%%%%%%%%%%%%%%%%%%%%%%%%%%
%%% LIVECOMS ARTICLE TEMPLATE
%%% ADAPTED FROM ELIFE ARTICLE TEMPLATE (8/10/2017)
%%%%%%%%%%%%%%%%%%%%%%%%%%%%%%%%%%%%%%%%%%%%%%%%%%%%%%%%%%%%
%%% PREAMBLE 
\documentclass[9pt]{livecoms}
% Use the 'onehalfspacing' option for 1.5 line spacing
% Use the 'doublespacing' option for 2.0 line spacing
% use the 'lineno' option for adding line numbers. 
% Please note that these options may affect formatting.

\usepackage[version=4]{mhchem} 
\usepackage{siunitx}
\DeclareSIUnit\Molar{M}
\newcommand{\versionnumber}{0.1}  % you should update the minor
                                  % version number in preprints and
                                  % major version number of
                                  % submissions.
\newcommand{\comm}[2]{{\color{red}#1: \emph{#2}}}
\newcommand{\todo}[1]{{\color{blue}TODO: \emph{#1}}}

%%%%%%%%%%%%%%%%%%%%%%%%%%%%%%%%%%%%%%%%%%%%%%%%%%%%%%%%%%%%
%%% ARTICLE SETUP
%%%%%%%%%%%%%%%%%%%%%%%%%%%%%%%%%%%%%%%%%%%%%%%%%%%%%%%%%%%%
\title{Best Practices for Physical Validation of Simulation Results
  (working title): v\versionnumber}

\author[1*]{Firstname Middlename Surname}
\author[1,2\authfn{1}\authfn{3}]{Firstname Middlename Familyname}
\author[2\authfn{1}\authfn{4}]{Firstname Initials Surname}
\author[2*]{Firstname Surname}
\affil[1]{Institution 1}
\affil[2]{Institution 2}

\corr{email1@example.com}{FMS}  % Correspondence emails.  FMS and FS are the appropriate authors initials. 
\corr{email2@example.com}{FS}

\contrib[\authfn{1}]{These authors contributed equally to this work}
\contrib[\authfn{2}]{These authors also contributed equally to this work}

\presentadd[\authfn{3}]{Department, Institute, Country}
\presentadd[\authfn{4}]{Department, Institute, Country}

%%%%%%%%%%%%%%%%%%%%%%%%%%%%%%%%%%%%%%%%%%%%%%%%%%%%%%%%%%%%
%%% ARTICLE START
%%%%%%%%%%%%%%%%%%%%%%%%%%%%%%%%%%%%%%%%%%%%%%%%%%%%%%%%%%%%

\begin{document}

\maketitle

\begin{abstract}
  \todo{Write an abstract.}
\end{abstract}


\section{Introduction}

\comm{PTM}{Make sure to have a look at the template file from
  LiveCoMS, they have some good examples. I defined two additional
  commands that I'd suggest to use:
  \begin{itemize}
  \item \texttt{\textbackslash{}comm\{Your Initials\}\{Put your
      comment here...\}} allows you to put comments directly in the
    LaTeX code (like the one you are just reading), which can easily
    be sorted out a later stage.
  \item \texttt{\textbackslash{}todo\{Write your todo here...\}} works
    in a similar way to add todo's in the text.
  \end{itemize}
  (By experience, comments and todos always pop up sooner or later
  when collaboratively editing a text, so better have a uniform style
  from the start...)}

\comm{PTM}{Another suggestion: I think it would be a good idea to make
  ample use additional files included via
  \texttt{\textbackslash{}input\{file.tex\}}, e.g. one for every
  section (or even subsection, if sections get very large). This keeps
  the main file tidy and should make merging significantly easier.}

\comm{PTM}{A last suggestion: Using the the \texttt{siunitx} package
  (which is included in the LiveCoMS template) makes typesetting of
  units painless.}

\section{Acknowledgments}

\todo{Funder and other information can be given here.}

\nocite{*} % This command displays all refs in the bib file
\bibliography{references}

%%%%%%%%%%%%%%%%%%%%%%%%%%%%%%%%%%%%%%%%%%%%%%%%%%%%%%%%%%%%
%%% APPENDICES
%%%%%%%%%%%%%%%%%%%%%%%%%%%%%%%%%%%%%%%%%%%%%%%%%%%%%%%%%%%%

\appendix
\begin{appendixbox}
\label{first:app}
\section{Glossary}
\todo{A glossary of technical terms used in the article.}

\end{appendixbox}

\end{document}
